% arara: lualatex: {shell: true}
% arara: latexmk: {clean: partial}
\documentclass{article}
%\usepackage[paperheight=\maxdimen,paperwidth=30cm,left=2cm,right=2cm,top=2cm,bottom=3cm]{geometry}%PACKAGE geometry
%\usepackage{varwidth}
%\usepackage[pdftex,active,tightpage]{preview}
%\renewcommand{\PreviewBorder}{1cm}

%\newcommand{\formatpage}{\leavevmode\setlength{\parskip}{\baselineskip}\setlength{\parindent}{0pt}}

%\newcommand{\StartPage}[1][\paperwidth]{\begin{preview}\begin{varwidth}{#1}\formatpage}%MACRO StartPage
%\newcommand{\EndPage}{\end{varwidth}\end{preview}}%MACRO EndPage

%\newcommand{\Newpage}[1][\paperwidth]{\EndPage\StartPage[#1]}%MACRO Newpage
%\newcommand{\page}[2][\paperwidth]{\StartPage[#1]#2\EndPage}

%\NewDocumentCommand{\doc}{O{\paperwidth}+m}{\StartPage[#1]#2\EndPage}

\usepackage{general}
\usepackage{caption}
\setcounter{secnumdepth}{5}

\titleformat{\subsubsection}[runin]
{\normalfont\normalsize\bfseries}{\thesubsubsection}{1em}{}

\title{Summary for Computational Number Theory at University of Minho}
\author{\alef}

\begin{document}
\StartPage[12cm]
	\maketitle

	Please contact me if you notice any mistaskes. This summary is not complete.

	\section{Topics worth learning}

	Theses topics had questions worth 3 points in past tests.

	\begin{enumerate}
		\item Use Fermat factorization
		\item Use $ρ$-Pollard
		\item Use $p-1$ Pollard

		\item Solve congruence equation knowing facts related to primitive root and index.

		\item Cipher a message using ElGamal + show that a number is a primitive root
		\item Calculate $φ(n)$ + decipher RSA

		\item Show there are not solutions for a congruence relation (quadratic residue)

		\item Euler pseudoprime
		\item Solovay-Strassen primality test
		\item Miller-Rabin primality test

		\item Suppose that if $n$ is a product of two primes. Show that factoring $n$ is equivalent to calculating $φ(n)$.

		\item Calculate Jacobi Symbol

	\end{enumerate}

	\section{Encryption Systems}

	\newcommand{\PuK}{\textrm{PubK}}
	\newcommand{\PrK}{\textrm{PrivK}}
	Let $M$ denote the message, $C$ the ciphertext.

		\subsection{ElGamal}

$\begin{array}{ll}
	\PrK &≡ 1<α<p-1\\
	\PuK &≡ (p∈ℙ,g:⟨g⟩ = ℤ_p^*,b≝g^α)\\
	C    &≡ (γ≝g^k,δ≝Mb^k),\quad k\textrm{ random element in } \{2,\dots,p-2\}\\
	M    &≡ δγ^{α^{-1}} \pmod p\\
\end{array}$


%	\begin{algorithm}[H]
%	\caption{ElGamal encipher}\label{alg:elgmal-enc}
%	\begin{algorithmic}
%		\REQUIRE $⟨g⟩ = ℤ_p^*$ \AND\\
%				 message $m$ \AND\\
%				 $b$ is the result of an exponentiation operation ($g^α$ where $1<α<p-1$ is a secret)
%		\ENSURE returns a ciphertext $(γ,δ)$ of the message.
%		\STATE $k \gets \textrm { random value in } \{2,\dots,p-2\}$
%		\STATE $γ \gets g^k \pmod p$
%		\STATE $δ \gets mb^k \pmod p$
%		\RETURN $γ,δ$
%	\end{algorithmic}
%\end{algorithm}
%
%	\begin{algorithm}[H]
%	\caption{ElGamal decipher}\label{alg:elgmal-dec}
%	\begin{algorithmic}
%		\REQUIRE cryptogram $(γ,δ)$ \AND
%				 $b$ is the result of an exponentiation operation ($r^α$ where $1<α<p-1$ is a secret)
%		\ENSURE returns the cyphertext deciphered, i.e., the message
%		\RETURN $δγ^{α^{-1}} \pmod p$
%	\end{algorithmic}
%\end{algorithm}

\subsection{RSA}

$\begin{array}{ll}
	\PrK &≡ d≝e^{-1} \pmod {φ(n) = (p-1)(q-1)}\\
	\PuK &≡ (n≝pq,e)\\
	C  	 &≡ M^e \pmod n\\
	M    &≡ C^d \pmod n
\end{array}$

	\section{Prime factorization}

	\subsection{Factoring given $φ(n)$}
	$\frac{-b+\sqrt{b^2 - 4n}}{2}$ where $b = n+1-φ(n)$ is a factor of $n$.

	\subsection{Fermat}

	\begin{algorithm}[H]
		\caption{Fermat factorization}\label{alg:fermat}
		\begin{algorithmic}
			\REQUIRE $n \textrm{ odd } ∈ ℕ$
			\ENSURE $(a+\sqrt{a^2-n})(a-\sqrt{a^2-n}) = n$
			\STATE $a \gets \sqrt{⌈n⌉}$
			\WHILE{$\sqrt{a^2-n} ∉ ℤ$}
			\STATE $a \gets a+1$
			\ENDWHILE
		\end{algorithmic}
	\end{algorithm}

	\subsection{$ρ$-Pollard}
	\begin{algorithm}[H]
		\caption{$ρ$-Pollard factorization}\label{alg:rho-pollard}
		\begin{algorithmic}
			\REQUIRE{b-smooth $g$, e.g. $g(x) = x^2+1$ \AND $x_0$, e.g., $x_0 ≝ 2$}
			\ENSURE{$\gcd(|x-y|,n)$ is nontrivial factor of $n$}
			\STATE $x \gets x_0$
			\STATE $y \gets x_0$
			\WHILE{$\gcd(|x-y|,n) = 1$}
			\STATE $x \gets g(x)$
			\STATE $y \gets g(g(y))$
			\ENDWHILE
		\end{algorithmic}
	\end{algorithm}

	\subsection{Pollard $p-1$}

%	\begin{algorithm}[H]
%		\caption{Pollard $p-1$}\label{alg:pollar-p-1}
%		\begin{algorithmic}
%			\REQUIRE $n \textrm{ odd composite } ∈ ℕ$
%			\STATE $M \gets \displaystyle\prod_{q ∈ ℙ≤B}{q^{⌊\log_qB⌋}}$
%			\LOOP
%			\STATE $x \gets 2^M-1 \pmod n$
%			\IF{$x = 0$}
%			\STATE continue
%			\ENDIF
%			\STATE $g \gets \gcd(x,n)$
%			\IF{$g ≠ 1$ \OR $g ≠ n$}
%			\RETURN g
%			\ENDIF
%			\ENDLOOP
%		\end{algorithmic}
%	\end{algorithm}

	The algorithm as presented by the professor

	\begin{algorithm}[H]
		\caption{Pollard $p-1$ simplified}\label{alg:pollar-p-1-simp}
		\begin{algorithmic}
			\REQUIRE $n \textrm{ odd composite } ∈ ℕ$
			\ENSURE $\gcd(r-1,n)$ is a nontrivial factor of $n$
			\STATE $r_0 \gets 2$
			\STATE $r \gets r_0$
			\WHILE{$\gcd(r-1,n) = 1$}
			\STATE $r \gets r*r_0 \pmod n$
			\ENDWHILE
		\end{algorithmic}
	\end{algorithm}

\section{Useful facts}

\subsection{Solving simple congruence equations knowing primitive roots}

Let $z_i ∈ ℤ$ and $I(a)$ be the the index with respect to a primitive root $g ∈ ℤ_n^*$.
To solve an equation of the type $$z_1x^{z_2} ≡ z_3 \pmod n$$
get $$I(z_1x^{z_2}) ≡ I(z_3) ⟺ z_2I(x) ≡ I(z_3) - I(z_1) \pmod {φ(n)}$$ to the form
$$I(x) ≡ I(z_4) \pmod {φ(n)}$$ then conclude $$x ≡ z_4 \pmod n$$

\subsection{Euler's theorem}
$$a ⟂ n ⟹ a^{φ(n)} ≡ 1 \bmod n$$

\subsection{Euler's totient function}

\subsubsection{Definition}
$$φ(n) ≝ n \displaystyle∏_{p|n} (1-\frac{1}{p})$$

\subsubsection{Useful facts}

\paragraph{multiplicative} $m ⟂ n ⟹ φ(mn) = φ(m)φ(n)$
\paragraph{prime power argument} $φ(p^k) = p^k - p^{k-1}$

\subsection{Reduced residue system}
$$\operatorname{RRS}(n) = R\textrm{ s.t. }
\begin{cases}
	∀r ∈ R.\quad \gcd(r,n)  = 1\\
	|R| = φ(n)\\
	∀r_1,r_2 ∈ R.\quad r_1 \not≡_n r_2
\end{cases}$$

\subsection{Primitive root modulo $n$}

\subsubsection{Definition}
$$g\textrm{ is primitive root modulo } n\textrm{ if and only if } ⟨g⟩ = ℤ^*_n$$

Alternatively, one can say $g$ is a primitive root of $n$ iff its order is $φ(n)$.

\subsubsection{Useful facts}
\paragraph{Fact} $⟨g⟩ = ℤ^*_p ⟹ g^{\frac{p-1}{2}} ≡ -1 \pmod p$

\paragraph{Condition for existence of primitive root}

$ℤ_n^*$ is cyclic iff $n$ is equal to $2,4,p^k,2p^k$ where $p^k$ is the power of and odd prime number. When (and only when) this group $ℤ_n^*$ is cyclic, a generator of this cyclic group is called a primitive root modulo $n$.

When $ℤ_n^*$ is non-cyclic, such primitive root elements mod $n$ do not exist.

\paragraph{Number of primitive roots}

The number of primitive roots modulo n, if there are any, is equal to $$φ(φ(n))$$

\subsubsection{Showing $g$ is primitive root of $n$}

$$∀i ∈ \{1,...,k\}.\quad g^{\frac{φ(n)}{p_i}} \not\equiv 1\bmod n ⟹ ⟨g⟩ = Z_n^*$$

Where $p_1,\dots,p_k$ are the different prime factors of $φ(n)$.

\subsection{Jacobi symbol}

\subsubsection{Definition}
The Jacobi symbol $(\frac{a}{n})$ is defined as the product of the Legendre symbol corresponding to the prime factors of $n$: $$(\frac{a}{n}) = (\frac{a}{p_1})^{α_1}(\frac{a}{p_2})^{α_2}\cdots(\frac{a}{p_k})^{α_k}$$

\subsubsection{Legendre symbol}
Definition of the Legendre symbol: $$(\frac{a}{p}) = \begin{cases}
	0  & \textrm{ if } a \equiv 0 \pmod p\\
	1  & \textrm{ if } a \not\equiv 0 \pmod p\textrm{ and for some integer } x: a\equiv x^2 \pmod p\\
	-1 & \textrm{ if } a \not\equiv 0 \pmod p\textrm{ and there is no such } x\\
	\end{cases}$$

\subsubsection{Useful properties}

\paragraph{Modular equivalence}
$$(\frac{a}{n}) = (\frac{b}{n}) \impliedby a ≡ b\pmod n$$
\paragraph{Coprimality}
$$(\frac{a}{n}) = 0 \impliedby \gcd(a,n) \not= 1$$
\paragraph{Multiplicative}
Completely multiplicative function (if fixing one of the arguments):\\
$$(\frac{ab}{mn}) = (\frac{a}{mn})(\frac{b}{mn}) = (\frac{ab}{m})(\frac{ab}{n})$$
\paragraph{Quadratic reciprocity}
Law of quadratic reciprocity: if $p$ and $q$ are odd positive coprime integers, then
$$(\frac{p}{q})(\frac{q}{p}) = (-1)^{\frac{p-1}{2}⋅\frac{q-1}{2}}$$
\paragraph{Euler's criteria}
${(\frac{a}{p} )}  ≡ a^{\frac{p-1}{2}} \pmod p$
\paragraph{Extra}
\begin{align*}
{(\frac{-1}{n})}  &= (-1)^{\frac{n-1}{2}}\\[.2cm]
{(\frac{2}{n} )}  &= (-1)^{\frac{n^2-1}{8}}\\[.2cm]
{(\frac{1}{n} )}  &= (\frac{n}{1}) = 1
\end{align*}

\section{Primality testing}

\subsection{Pseudoprimes}

\subsubsection{Weak pseudoprime}
A composite number $n$ such that $b^n ≡ b \pmod n$ is a weak pseudoprime to base $b$.

\subsubsection{Strong pseudoprime}
A composite number $n$ such that it passes the Miller-Rabin test for base $b$.

\subsubsection{Euler pseudoprime}
An odd composite integer $n$ is called an Euler pseudoprime to base $b$, if
$$b^{\frac{n-1}{2}} ≡ (\frac{a}{n}) \pmod n$$

\subsubsection{Fermat pseudoprime}
A composite integer $n$ is called a Fermat pseudoprime to base $b>1$ if
$$b^{n-1} ≡ 1 \pmod n$$

\subsubsection{Carmichael number}
$n$ is a Carmichael number if it's a Fermat pseudoprime for all values $b$ coprime to $n$.

\subsection{Solovay-Strassen}
$(\frac{b}{n}) = 0 ∨ (\frac{b}{n}) \not\equiv b^{\frac{n-1}{2}} \bmod n ⟹ n\textrm{ is not prime}$

\subsection{Miller-Rabin}
Let $n - 1 = 2^ed$ with $n,d$ odd.

Let $\gcd(1<b<n,n) = 1$.

If $b^d ≡ 1 \bmod n$ or $\displaystyle∃_{0≤j<e}.\ b^{2^{j}d} ≡ -1 \bmod n$, then $n$ passes the test for base $b$.

If $n$ is composite, the probability that $n$ passes the test for $k$ bases is
$< \frac{1}{4k}$.
\EndPage
\end{document}
